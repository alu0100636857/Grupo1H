%%%%%%%%%%%%%%%%%%%%%%%%%%%%%%%%%%%%%%%%%%%%%%%%%%%%%%%%%%%%%%%%%%%%%%%%%%%%%
% Chapter 1: Motivaci�n y Objetivos 
%%%%%%%%%%%%%%%%%%%%%%%%%%%%%%%%%%%%%%%%%%%%%%%%%%%%%%%%%%%%%%%%%%%%%%%%%%%%%%%


%---------------------------------------------------------------------------------
\section{Motivaci�n}
\label{1:sec:1}
\begin{itemize}
  \item	 Familiarizaci�n con la creaci�n de informes cient�ficos.
	Esta tarea nos aportar� unas nociones b�sicas que nos ayudar�n en el transcurso de nuestra carrera. Ser� importante su correcto aprendizaje para el futoro, esto se debe a que jugar� un papel importante en el trabajo de fin de grado.
  \item  Correcto aprendizaje de diversas herramientas:
	\begin{itemize}
	 \item	\LaTeX{}
	 \item	Beamer
	 \item	Python
	\end{itemize}
\end{itemize}


%---------------------------------------------------------------------------------
\section{Objetivos}
\label{1:sec:2}
  El trabajo se ha realizado con los objetivos siguientes:

\begin{itemize}
  \item  Aprender a buscar informaci�n adecuada para el desarrollo de la tarea a realizar. Utilizando, entre otras cosas, las herramientas de la BULL.
  \item  Analizar el problema a estudiar y dise�ar una soluci�n a dicho problema.
  \item  Crear e implementar un algoritmo de resoluci�n al problema dado mediante Python.
  \item  Verificar mediante pruebas que la soluci�n propuesta es correcta y eficiente.
  \item  Elaborar un informe final sobre el tema utilizando \LaTeX{}
  \item  Realizar una presentaci�n p�blica del mismo.
\end{itemize}

